\documentclass{article}
\usepackage{mathtools}
\usepackage[top=1in, bottom=1in, left=1in, right=1in]{geometry}
\usepackage{graphicx}
\usepackage[normalem]{ulem}
\usepackage{fancyhdr}
\usepackage{enumerate}
\usepackage{python}
\usepackage{verbatim}
\pagestyle{fancyplain}


\lhead{CECS 529}

\rhead{A. Shawn Bandy 
(003635396)}


\begin{document}
\title{Lecture 1 Exercises}
\author{A. Shawn Bandy}
\date{August 31\textsuperscript{st}, 2013}
\maketitle

\begin{enumerate}
\item Consider the following document collection:
\begin{itemize}
\item \textbf{Doc 1:} breakthrough drug for schizophrenia
\item \textbf{Doc 2:} new schizophrenia drug
\item \textbf{Doc 3:} new approach to treatment of schizophrenia
\item \textbf{Doc 4:} new hope for schizophrenia patients
\end{itemize}
\begin{python}[ex1lib.py]
import shelve 
import ex1lib

documents = {0: 'breakthrough drug for schizophrenia',1: 'new schizophrenia drug',2: 'new approach to treatment of schizophrenia',3: 'new hope for schizophrenia patients'}

ex1lib.print_documents(documents)

s = shelve.open('ex1');
try:
	s['documents'] = documents
finally:
	s.close()



\end{python}
Provide both the term-document matrix and inverted index for this document collection.
What documents will be returned for the Boolean query \texttt{schizophrenia} AND
\texttt{drug}? For the query \texttt{for} AND NOT(\texttt{drug} or \texttt{approach})?

\begin{python}
import shelve
import ex1lib

s = shelve.open('ex1');
try:
        documents = s['documents']
finally:
        s.close

m = ex1lib.calc_term_document_matrix(documents)
(header,firstline,rows,footer) = ex1lib.format_term_document_matrix(m,documents)
print r"Hello"
print header
print firstline
for word in rows:
	print word + "&" + rows[word]
print footer
\end{python}



\item Suppose that $x$ and $y$ are the respective postings-list sizes for the terms
\texttt{Romney} and \texttt{Obama}. Assuming $n$ documents in the collection.
What are the worst-case running times for returning the list of documents that satisfy 
\texttt{Romney} AND NOT \texttt{Obama}? \texttt{Romney} OR NOT \texttt{Obama}? 
Explain.

\item Convert
\begin{verbatim}
(a or b) and not(c or d)
\end{verbatim} 
to disjunctive normal form. If a, b, c, and d are terms with respective postings-list sizes 
$x$, $y$, $z$, and $w$, compare the worst-case running times of the original query
with the DNF query. Assume $n$ documents in the entire collection.

\item Give an example where a,b, and c are terms with respective postings-list sizes
$x < y < z$, and for which 
\begin{verbatim}
(a and b) and c
\end{verbatim} 
will have a longer running time than
\begin{verbatim}
(c and b) and a
\end{verbatim}

\item Use the Integral Theorem to obtain the order of growth of the sequence
\[1+1/\sqrt{2}+1/\sqrt{3}+\cdots.\]

$\sum_{i=1}^m f(i) = \theta (integral f(x) from 1 to m dx)$

\item Repeat Example 5, but now use $c=0.5$, followed by $c=2$. Compare the memory usage
for both cases with the $c=1$ case computed in Example 5.

%t_i% has approximately floor n/i^s posting, for s = 2 and s = 0.5

\item Consider a collection of $n$ documents and $m$ terms, where, for each document $d$ and term $t$, $t$ appears
at most once in $d$. Moreover, if $t_{1},\ldots,t_{m}$ is a listing of terms in decreasing order of frequency, assume that
$t_{i}$ occurs in $1/i$ of the documents. a) Give a big-$\Theta$ estimate of the number of tokens contained in this collection. b) What percentage of the most frequently occurring terms account for 80\% of these tokens? Note: Pareto's
rule would imply an answer of 20\%. Is this correct?

T =  of 1's in term\_doc matrix

T = n + n/2 + n/3 ... n/m

m is arbitrary

m = sqrt(T) = sqrt(n log m) what does n have to be to make this an actual equality?

\item For the document collection of the previous problem, how would $n$ have to depend on $m$ in order for Heap's Law
to be obeyed. In other words, what should we replace $n$ by in terms of $m$ so that the square root of the number of
tokens is on the order of $m$? 



\item Try the following Google queries: burglar, burglar burglar, and 
burglar OR burglar. Look at the estimated number of results and top hits.
Are they identical (as they should be)?

\item Try the Google queries knight, conquer, and knight OR conquer.
Are the number of hits for the third query bounded by the sum of the number 
of hits for the first two queries (as they should be)?

\end{enumerate}

\end{document}